\documentclass{article}
\usepackage{enumerate}
\usepackage{graphicx}
\usepackage{float}

\usepackage{amsmath}
\usepackage[margin=1in]{geometry}
\usepackage[parfill]{parskip}

\newcommand{\heading}[1]{\bigskip \textbf{#1}}
\DeclareMathOperator{\sech}{sech}

\title{Physics 112 Problem Set 4 \\ \large{Holzapfel, Section 102}}
\author{Sahil Chinoy}
\date{October 2, 2017}

\begin{document}
\maketitle{}

\begin{enumerate}

	\item 

	\begin{enumerate}[(a)]

		\item

		Starting from the spectral intensity per unit wavelength,

		$$I_\lambda = \frac{2 h \nu^5}{c^3 (\exp(h\nu / kT) - 1)},$$

		we substitute $x = h \nu / kT$ and differentiate to find the frequency at which the intensity (per wavelength) peaks

		$$0 = \frac{d I_\lambda}{d x} \implies e^x(5-x) = 5 \implies x = 4.965.$$

		This enables us to calculate the temperature of the sun

		$$T_S = \frac{h \nu_{peak}}{4.965 \times k} =  \frac{hc}{4.965 \times k \lambda_{peak}} = 6037 \text{ K.}$$

		Now, for a cone of apex angle $2\theta$, the solid angle is

		$$\Omega = \int \limits_0^{2\pi} \int \limits_0^\theta \sin \theta \; d\theta d\phi = 2\pi (1-\cos\theta) \approx \pi \theta^2$$

		for $\theta \ll 1$. So the proportion of solar flux that reaches the Earth is $\pi \theta^2 / 4 \pi = \theta^2 / 4$, where $2\theta$ is the angle subtended by the sun, $0.5^\circ$.

		Since the flux emitted by the sun must be proportional to the fourth power of its temperature (by the Stefan-Boltzmann law) and the flux absorbed by the Earth must likewise be proportional to the fourth power of its temperature, we have

		$$T_E^4 = \frac{\theta^2}{4} T_S^4$$

		or

		$$T_E = \sqrt{\frac{(0.5^\circ / 2) \times (\pi / 180^\circ)}{2}} T_S = 282 \text{ K.}$$

		This is approximately $9^\circ$ C, which seems far too low, especially since the sun is not illuminating the Earth at all hours of the day.

		\item

		If we assume that all the radiation from the sun absorbed by the Earth is re-emitted and absorbed by the atmosphere, and that the atmosphere subsequently re-emits this radiation equally in both directions (towards and away from the Earth), then the flux at the surface of the Earth is $1.5 \times$ the flux emitted by the sun that reaches the Earth, which gives

		$$T_E = \left( \frac{3}{2} \right)^{1/4} \sqrt{\frac{(0.5^\circ / 2) \times (\pi / 180^\circ)}{2}} T_S = 312 \text{ K,}$$

		or about $40^\circ$ C. 

	\end{enumerate}

	\item

	\begin{enumerate}

		\item

		The multiplicity is

		$$g(N_2) = \frac{N!}{N_2! (N-N_2)!}.$$

		\item

		The entropy is

		$$\sigma = \log g = \log (N!) - \log (N_2!) - \log((N-N_2)!),$$

		and using the Stirling approximation for large $N$,

		\begin{gather*}
		\sigma = N \log N - N - (N_2 \log N_2 - N_2 + (N-N_2) \log(N-N_2) - (N-N_2)) \\
		\sigma = N\log N - N_2 \log N_2 - (N-N_2)\log(N-N_2).
		\end{gather*}

		So

		\begin{gather*}
		\frac{\partial \sigma}{\partial N_2} = - \log N_2 - 1 + 1 + \log(N-N_2) = \log \left( \frac{N-N_2}{N_2} \right) = \log\left( \frac{N}{N_2} - 1 \right)
		\end{gather*}

		But $U = N_2\Delta$, so $\partial \sigma / \partial U = (1/\Delta) (\partial \sigma / \partial N_2),$ and since $1/\tau = \partial \sigma / \partial U$, 

		\begin{gather*}
		\frac{1}{\tau} = \frac{1}{\Delta} \log\left( \frac{N}{N_2} - 1 \right) \\
		\exp{\left(\frac{\Delta}{\tau} \right)} + 1 = \frac{N}{N_2} \\
		N_2 = \frac{N}{1 + \exp(\Delta/\tau)},
		\end{gather*}

		thus 

		$$U = \frac{N \Delta}{1 + \exp(\Delta/\tau)}.$$

		In the limit $\tau \to 0$, $U \to 0$, as expected. In the limit $\tau \to \infty$, $U \to N\Delta / 2$, which implies that half the systems are in the excited state.

		\item

		For the canonical ensemble, the partition function for a single system is

		$$Z_1 = \sum \limits_s E_s \exp(- E_s/\tau) = 1 + \exp(- \Delta / \tau).$$

		Thus the expected energy for a single system is 

		$$U_1 = \frac{\Delta \exp(-\Delta / \tau)}{1 + \exp(- \Delta / \tau)} = \frac{\Delta}{1 + \exp(\Delta / \tau)},$$

		and the expected energy for $N$ identical systems is

		$$U = \frac{N \Delta}{1 + \exp(\Delta / \tau)},$$

		which agrees with the result from the microcanonical ensemble.


	\end{enumerate}

	\item

	\begin{enumerate}

		\item

		A single mode, indexed by $n$, at frequency $\omega_n$, with $s$ photons has energy $\epsilon_n = s \hbar \omega_n$. Thus the partition function for a single mode is

		$$Z_n = \sum \limits_{s=0}^{\infty} \exp \left( -s \hbar \omega_n / \tau \right) = \left[ 1 - \exp \left( - \hbar \omega_n / \tau \right) \right]^{-1}.$$

		Then the partition function for the gas with $n$ modes is 

		$$Z = \prod\limits_{n} \left[ 1 - \exp \left( -\hbar \omega_n / \tau \right) \right]^{-1}.$$

		The Helmholtz free energy is then

		$$F = -\tau \log Z = \tau \sum \limits_{n=0}^\infty  \log \left[ 1 - \exp \left( -\hbar \omega_n / \tau \right) \right].$$

		To convert a sum over modes into an integral over frequencies, we need the density of photons at each frequency,

		$$\rho_\omega d\omega = \frac{V \omega^2}{c^3 \pi^2} d\omega,$$

		thus

		$$F = \frac{V \tau}{c^3 \pi^2} \int \limits_0^\infty \omega^2 \log \left[ 1 - \exp \left( - \hbar \omega / \tau \right) \right] \; d\omega.$$

		Substituting $x = \hbar \omega / \tau$, 

		\begin{gather*}
		F = \frac{V \tau}{c^3 \pi^2} \left( \frac{\tau}{\hbar} \right)^3 \int \limits_0^\infty x^2 \log (1 - e^{-x}) \; dx = \frac{ V \tau^4}{ c^3 \hbar^3 \pi^2} \left[ \left. \frac{1}{3}\log(1 - e^{-x} )  x^3 \right\rvert_0^\infty - \frac{1}{3} \int\limits_0^\infty \frac{x^3 e^{-x}}{1 - e^{-x}} \; dx \right] \\
		F = - \frac{ V \tau^4}{3  c^3 \hbar^3 \pi^2} \int\limits_0^\infty \frac{x^3}{e^{x} - 1} \; dx = - \frac{ V \tau^4}{3  c^3 \hbar^3 \pi^2}  \frac{\pi^4}{15} = - \frac{ V \tau^4 \pi^2}{45  c^3 \hbar^3},
		\end{gather*}

		where I've looked up the value of the dimensionless integral.

		\item

		The radiation pressure is then

		$$p = - \frac{\partial F}{\partial V} = \frac{ \tau^4 \pi^2}{45  c^3 \hbar^3} = \frac{u}{3},$$

		where $u$ is the energy density of the photon gas, $u = \tau^4 \pi^2 / 15  c^3 \hbar^3$. Compare this to the monatomic ideal gas, where $p = 2u/3$. The radiation pressure is greater for the ideal gas than the photon gas.

	\end{enumerate}

	\item

	\begin{enumerate}

		\item

		The flux incident on the middle plane is $J_u + J_l$, and the flux emitted from the middle plane is $2J_m$, so 

		$$\sigma_B(T_u^4 + T_l^4) = 2 \sigma_B T_m^4,$$

		or 

		$$T_m = \left[ \frac{1}{2}(T_u^4 + T_l^4) \right]^{1/4}.$$

		Now, the flux between the upper plane and the middle plane is

		$$J_{net} = J_u - J_m = \sigma_B T_u^4 - \frac{\sigma_B}{2} \left( T_u^4 + T_l^4 \right) = \frac{\sigma_B}{2} \left( T_u^4 - T_l^4\right),$$

		which is half of the flux without the middle plane. Likewise, the flux between the middle plane and the lower plane is

		$$J_{net} = J_m - J_l =  \frac{\sigma_B}{2} \left( T_u^4 + T_l^4 \right) - \sigma_B T_l^4 - = \frac{\sigma_B}{2} \left( T_u^4 - T_l^4\right).$$

		\item

		The radiation from the upper surface will bounce back and forth, and with each successive reflection, the flux will be reduced by a factor of $(1 - \epsilon)$. However, since the radiation alternates direction as it bounces back and forth, the even-indexed terms will contribute positive flux and the odd-indexed terms will contribute negative flux, so the net flux due to this surface is

		$$F_u = \sigma_B T_u^4 \left[ \epsilon - \epsilon(1-\epsilon) + \epsilon (1-\epsilon)^2 - \ldots \right] = \epsilon \sigma_B T_u^4 \sum \limits_{n=0}^\infty (-1)^n(1-\epsilon)^n = \left( \frac{\epsilon}{2-\epsilon} \right) \sigma_B T_u^4.$$

		Symmetrically, for the lower surface, only the odd-indexed terms will contribute positive flux. So the net flux due to this surface is

		$$F_l = \sigma_B T_l^4 \left[ -\epsilon + \epsilon(1-\epsilon) - \epsilon (1-\epsilon)^2 + \ldots \right] = -\epsilon \sigma_B T_l^4 \sum \limits_{n=0}^\infty (-1)^n(1-\epsilon)^n = -\left( \frac{\epsilon}{2-\epsilon} \right) \sigma_B T_l^4.$$

		Thus, the net flux in the middle region is

		$$F = \left( \frac{\epsilon}{2-\epsilon} \right) \sigma_B (T_u^4 - T_l^4).$$

	\end{enumerate}

	\item

	\begin{enumerate}

		\item

		Consider photon modes indexed by $j$. Each mode has $s_j$ photons of energy $\hbar \omega_j$. Then the total energy is

		$$U = \sum \limits_j s_j \hbar \omega_j,$$

		thus the pressure is

		$$p = -\frac{\partial U}{\partial V} = - \sum \limits_j s_j \hbar \frac{\partial \omega_j} {\partial V}.$$

		Consider a cube of side length length $L$, volume $L^3$. Imposing periodic boundary conditions, a half-integer number of wavelengths must fit in the side length, so $\lambda_j = 2L / j$, so the (angular) frequency is $\omega_j = 2\pi c / \lambda_j = j\pi c/ L = j \pi c V^{-1/3}.$ Then

		$$\frac{\partial \omega_j}{\partial V} = -\frac{j \pi c}{V^{4/3}} = - \frac{\omega_j}{3V}.$$

		The pressure is then 

		$$p = \frac{1}{3V} \sum \limits_j s_j \hbar \omega_j = \frac{U}{3V} = \frac{u}{3},$$

		as previously calculated in problem 3(b).

		\item

		See 3(c).

	\end{enumerate}

	\item

	\begin{enumerate}

		\item

		Starting from the fundamental thermodynamic relation $dU = \tau d\sigma - p dV$, we divide by $d \tau$. Holding $V$ constant implies $dV = 0$, thus

		$$ \left( \frac{\partial U}{ \partial \tau} \right)_V = \tau \left( \frac{\partial \sigma}{\partial \tau} \right)_V,$$

		or $C_V / \tau = (\partial \sigma / \partial \tau)_V.$

		\item

		$$\left( \frac{\partial \sigma}{\partial V} \right)_\tau = - \frac{\partial^2 F}{\partial \tau \partial V},$$

		and 

		$$\left( \frac{\partial p}{\partial \tau} \right)_V = - \frac{\partial^2 F}{\partial V \partial \tau},$$

		thus

		$$\left( \frac{\partial \sigma}{\partial V} \right)_\tau  = \left( \frac{\partial p}{\partial \tau} \right)_V.$$

		\item

		Starting from

		$$\frac{\partial^2 \sigma}{\partial V \partial \tau} = \frac{\partial^2 \sigma}{\partial \tau \partial V},$$

		we see that

		$$\frac{\partial^2 \sigma}{\partial \tau \partial V} = \frac{\partial}{\partial \tau} \left( \frac{\partial \sigma}{\partial V} \right)_\tau = \frac{\partial^2 p}{\partial \tau ^2} = \frac{1}{3} \frac{\partial ^2u}{\partial \tau^2},$$

		since $p = u/3$. Also,

		$$\frac{\partial^2 \sigma}{\partial V \partial \tau} = \frac{\partial}{\partial V} \left( \frac{\partial \sigma}{\partial \tau} \right)_V = \frac{\partial}{\partial V}  \left( \frac{C_V}{\tau} \right) = \frac{1}{\tau} \frac{\partial^2 U}{\partial V \partial \tau} = \frac{1}{\tau} \frac{\partial u}{\partial \tau},$$

		since $\partial U / \partial V = u$. Now $u$ is independent of volume, so we can write this as a second-order ordinary differential equation

		$$\frac{\partial u}{\partial \tau} = \frac{\tau}{3} \frac{\partial^2 u}{\partial \tau^2}.$$

		We see that $u(\tau) \sim \tau^4$ is a solution to this differential equation, as the Stefan-Boltzmann law shows.

	\end{enumerate}

\end{enumerate}

\end{document}