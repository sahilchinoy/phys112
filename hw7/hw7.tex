\documentclass{article}
\usepackage{enumerate}
\usepackage{graphicx}
\usepackage{float}

\usepackage{amsmath}
\usepackage[margin=1in]{geometry}
\usepackage[parfill]{parskip}

\newcommand{\heading}[1]{\bigskip \textbf{#1}}
\DeclareMathOperator{\sech}{sech}

\title{Physics 112 Problem Set 7 \\ \large{Holzapfel, Section 102}}
\author{Sahil Chinoy}
\date{October 25, 2017}

\begin{document}
\maketitle{}

\begin{enumerate}

	\item

	\begin{enumerate}[(a)]

		\item

		Defining the absolute activity $\lambda = \exp(\mu/\tau)$, the Gibbs sum is

		$$\mathcal{Z} = \sum_{ASN} \exp \left( \frac{N\mu - \epsilon}{\tau} \right) = 1 + \lambda \exp ( - \epsilon / \tau) + \lambda^2 \exp ( - 2\epsilon / \tau ).$$

		Then the average occupancy is 

		$$\langle N \rangle = \lambda \frac{d}{d\lambda} (\log \mathcal{Z}) = \frac{\lambda \exp(-\epsilon/\tau) + 2\lambda^2 \exp(-2\epsilon/\tau)}{\mathcal{Z}}.$$

		\item

		Now, since the state with energy $\epsilon$ is doubly degenerate, the Gibbs sum is

		$$\mathcal{Z} =  1 + 2 \lambda \exp ( - \epsilon / \tau) + \lambda^2 \exp ( - 2\epsilon / \tau ).$$

		Thus

		$$\langle N \rangle = \lambda \frac{d}{d\lambda} (\log \mathcal{Z}) = \frac{2\lambda \exp(-\epsilon/\tau) + 2\lambda^2 \exp(-2\epsilon/\tau)}{\mathcal{Z}}.$$


	\end{enumerate}

	\item

	For a single particle, $N=1$, so we will calculate the partition function $Z$, from which we can calculate the mean energy $\langle U \rangle$.

	$$Z = \sum_s \exp(-\epsilon_s / \tau) = \sum_s \exp(-p_sc / \tau).$$

	To convert the sum to an integral, we need an expression for the density of states in momentum space (this is valid since we are in the high-energy limit, $pc \gg Mc^2$)

	$$D(p) \; dp = \frac{1}{8} 4\pi p^2 dp = \frac{1}{2} \pi p^2 \; dp$$

	where the factor of $1/8$ comes from integrating over the positive portion of $p$-space (note that we've assumed spherical symmetry). Then the partition function becomes

	$$Z = \frac{\pi}{2} \int \limits_0^\infty p^2 \exp(-pc / \tau) dp = \frac{\pi \tau^3}{2c^3} \int \limits_0^\infty x^2 \exp(-x) dx = \frac{ \pi \tau^3}{c^3}.$$

	Then the average energy is

	$$\langle U \rangle = \tau^2 \frac{\partial \log Z}{\partial \tau} = \tau^2 \frac{c^3}{\pi \tau^3} \frac{ 3 \pi \tau^2}{c^3} = 3 \tau.$$

	\item

	\begin{enumerate}[(a)]

		\item

		The pressure of an individual state with energy $\epsilon_s$ is $-(\partial \epsilon_s / \partial V)_N$, so the expected pressure, averaging over all states and normalizing by the partition function, is

		$$p = -\frac{1}{Z} \sum_s \left( \frac{\partial \epsilon_s}{ \partial V} \right)_N \exp(-\epsilon_s / \tau).$$

		\item

		For an ideal gas, 

		$$\epsilon_s = \frac{\hbar^2}{2M} \left( \frac{\pi n}{V^{1/3}} \right)^2$$

		so

		$$\left( \frac{\partial \epsilon_s}{ \partial V} \right)_N = -\frac{2}{3} \frac{\hbar^2}{2M} \frac{(\pi n)^2}{V^{5/3}} = -\frac{2}{3}\frac{\epsilon_s}{V}.$$

		\item

		Using the result from (a), the pressure is

		$$p = \frac{2}{3V} \frac{1}{Z} \sum_s  \epsilon_s \exp(-\epsilon_s / \tau) = \frac{2U}{3V}.$$

		\item

		We know that at $T =0$, the internal energy is $U = 3N \epsilon_F / 5$, where $\epsilon_F$ is the Fermi energy (Kittel 7.10). So $p = 2n \epsilon_F / 5$ at $T =0$. Also, $\epsilon_F = (\hbar^2 / 2m) (3 \pi^2 n)^{2/3}$ (Kittel 7.7).

		With $n = 8 \times 10^{22} \text{ cm}^{-3}$ and $m_e = 9.11 \times 10^{-31}$ kg, this evaluates to $p = 3.4 \times 10^5$ atm.

	\end{enumerate}

	\item

	\begin{enumerate}[(a)]

		\item

		The total number of atoms is the sum of the average number of atoms in each orbital, so, defining $\lambda = \exp(\mu / \tau)$

		$$N = \lambda \sum_s \exp(-\epsilon_s / \tau)$$.

		For the states with lower internal energy, $\epsilon_s$ is the same as for an ideal monatomic gas with one internal energy state, so $\sum_s \exp(-\epsilon_s / \tau) = Z_1 = V n_Q$. For the states with greater internal energy, $\epsilon_{s'} = \epsilon_s + \Delta$, so

		$$\sum_{s} \exp \left( \frac{-\epsilon_s - \Delta}{ \tau} \right) = \exp(-\Delta/\tau) Z_1 = \exp(-\Delta/\tau) Vn_Q.$$

		Then

		$$N = \lambda Vn_Q (1 + \exp(-\Delta/\tau))$$

		so

		$$\lambda = \frac{n}{n_Q} \frac{1}{1 + \exp(-\Delta/\tau)}$$

		and

		$$\mu = \tau \log \left( \frac{n}{n_Q} \right) - \tau \log \left(1 + \exp \left( \frac{-\Delta}{\tau} \right) \right).$$

		\item

		The partition function for a single atom is

		$$Z_1 = \sum_s \exp(-\epsilon_s / \tau)$$

		so by the same logic as in (a)

		$$Z_1 = Vn_Q (1 + \exp(-\Delta/\tau)).$$

		Then for $N$ atoms,

		$$Z = \frac{Z_1^N}{N!} = \frac{(Vn_Q)^N}{N!} \left(1 + \exp \left( \frac{-\Delta}{\tau} \right) \right)^N.$$

		Thus

		\begin{gather*}
		F = -\tau \log Z = -\tau N \log (V n_Q) + \tau N \log N - \tau N - \tau N \log(1 + \exp(- \Delta / \tau)) \\
		F = \tau N (\log (n / n_Q) - 1) - \tau N \log(1 + \exp(- \Delta / \tau)) \\
		F = F_{ideal} - \tau N \log(1 + \exp(- \Delta / \tau))
		\end{gather*}

		where $F_{ideal}$ is the free energy for the ideal gas with one internal energy state.

		\item

		The entropy is given by

		$$\sigma = - \left( \frac{\partial F}{\partial \tau} \right)_V$$

		and recall that the entropy for an ideal gas with one internal energy state is given by $\sigma_{ideal} = N (\log (n_Q / n) + 5/2)$. Thus

		\begin{gather*}
		\sigma = N \left( \log \left( \frac{n_Q}{n} + \frac{5}{2} \right) \right) + N \log \left( 1 + \exp \left( \frac{- \Delta}{\tau} \right) \right) + N \tau \frac{\exp(-\Delta/\tau)}{1 + \exp(-\Delta /\tau)} \frac{\Delta}{\tau^2} \\
		\sigma = N \left( \log \left( \frac{n_Q}{n} + \frac{5}{2} \right) \right) + N \log \left( 1 + \exp \left( \frac{- \Delta}{\tau} \right) \right) + \frac{N \Delta}{\tau} \frac{1}{1 + \exp(\Delta /\tau)}.
		\end{gather*}

		\item

		The pressure is the same as for an ideal gas with one internal energy state,

		$$p = - \left( \frac{\partial F}{\partial V} \right)_\tau = \frac{N\tau}{V},$$

		since we have not introduced any new terms with volume dependence.

		\item

		The heat capacity at constant pressure is

		\begin{gather*}
		C_p = \tau \left( \frac{\partial \sigma}{\partial T} \right)_p = k_B \tau  \left( \frac{\partial \sigma}{\partial \tau} \right)_p \\
		C_p = k_B \tau N \frac{\partial }{\partial \tau}\left( \log (n_Q) - \log(n) \right) + Nk_b \tau  \frac{\Delta}{\tau^2}\frac{\exp(-\Delta/\tau)}{1 + \exp(-\Delta/\tau)} \\
		- Nk_B \tau \frac{\Delta}{\tau^2} \frac{1}{1 + \exp(\Delta /\tau)} + Nk_B \tau \frac{\Delta}{\tau} \frac{\Delta}{\tau^2} \frac{\exp(\Delta /\tau)}{(1 + \exp(\Delta /\tau))^2} \\
		C_p = Nk_B\tau \left( \frac{3}{2 \tau} - \frac{\partial}{\partial \tau} \log \left( \frac{p}{\tau} \right) \right) + Nk_B \frac{\Delta^2}{\tau^2} \frac{\exp(\Delta /\tau)}{(1 + \exp(\Delta /\tau))^2} \\
		C_p = \frac{5}{2}N k_B + Nk_B \frac{\Delta^2}{\tau^2} \frac{\exp(\Delta /\tau)}{(1 + \exp(\Delta /\tau))^2}.
		\end{gather*}

	\end{enumerate}

	\item

	\begin{enumerate}[(a)]

		\item

		Starting from $N = \lambda \sum_s \exp(-\epsilon_s / \tau)$, we know that in the classical limit, we can approximte the sum with an integral. Each momentum state $dp$ takes up a de Broglie wavelength of $h / dp$, and thus occupies an area of $ (h / dp)^2.$ Thus the density of states is $D(p) dp = A (dp)^2 / h^2$, and assuming momentum space is circularly symmetric, $(dp)^2 = 2 \pi p dp$. Also, $p^2 = 2m E$ implies $2p dp = 2m dE$, so

		$$D(E)dE = \frac{2 \pi A m}{h^2} dE.$$

		then

		$$N = \lambda  \sum_s \exp(-\epsilon_s / \tau) = \lambda \frac{2 \pi A m}{h^2} \int \limits_0^\infty \exp(-\epsilon/\tau) d\epsilon = \lambda \frac{2 \pi A m}{h^2} \tau.$$

		Since $\lambda = \exp(\mu / \tau)$, this means

		$$\mu = \tau \log \left( \frac{N h^2}{2 \pi A m \tau}\right).$$

		\item

		We know that each degree of freedom in an atom of three-dimensional ideal gas carries $\tau / 2$ energy, for a total energy of $U = 3N \tau/2$. By analogy, for a two-dimensional gas, $U = N \tau.$

		\item

		The entropy can be calculated from the free energy. From (a), the partition function is

		$$Z_1 = \frac{2\pi A m \tau}{h^2}$$

		and $Z = Z_1^N / N!$, so

		\begin{gather*}
		F = - \tau \log Z = -N \tau \log \left( \frac{2\pi A m \tau}{h^2} \right) + N \tau \log N - N \tau \\
		F = -N\tau \left( \log \left( \frac{2\pi A m \tau}{N h^2} \right) + 1 \right).
		\end{gather*}

		Thus 

		$$\sigma = - \left( \frac{\partial F}{\partial \tau} \right)_V = N \left(  \log \left( \frac{2\pi A m \tau}{N h^2} \right) + 1 \right) + N\tau \frac{1}{\tau} =  N \left(  \log \left( \frac{2\pi A m \tau}{N h^2} \right) + 2 \right).$$

	\end{enumerate}

	\item

	\begin{enumerate}[(a)]

		\item

		Since both electrons and holes have degeneracy 2, and the energy gap is $E_g$, the equilibrium condition $\mu_e + \mu_h = 0$ implies $\log (n_e / 2n_{Q,e}) + \log(n_h / 2n_{Q,h} \exp(-E_g/\tau)) =0$, or

		\begin{gather*}
		n_e n_h = (2n_{Q,e})(2n_{Q,h}) \exp(-E_g/\tau) \\
		n_e n_h = 4 (m_e m_h)^{3/2} \left( \frac{\tau}{2 \pi \hbar^2} \right)^3 \exp(-E_g/\tau).
		\end{gather*}

		In equilibrium, the density of holes and electrons needs to be equal (or more phonons would be produced), so $n_e = n_h$, and

		$$n_e = 2 \left( \frac{\tau \sqrt{m_e m_h}}{2 \pi \hbar^2} \right)^{3/2} \exp(-E_g/2\tau).$$

		\item

		The chemical potential $\mu_e = \tau \log(n_e / 2n_{Q,e})$, so

		\begin{gather*}
		\mu_e = \tau \log 2 + \frac{3}{2}\tau \log \left( \frac{\tau}{2 \pi \hbar^2} \right) + \frac{3}{2}\tau \log(\sqrt{m_em_h}) - \frac{E_g}{2} - \tau \log 2 - \frac{3}{2}\tau \log \left( \frac{m_e \tau}{2 \pi \hbar^2} \right) \\
		\mu_e = \frac{3}{4}\tau \log \left( \frac{m_h}{m_e} \right) - \frac{E_g}{2}.
		\end{gather*}

		\item

		The equilibrium condition reflects just the chemical potential due to the free electrons, since the electrons from donor atoms are not part of the ``gas," so $\log ((n_e' - n_d) / 2n_{Q,e}) + \log(n_h' / 2n_{Q,h} \exp(-E_g/\tau)) =0$, or

		$$
		(n_e' - n_d) n_h' = (2n_{Q,e})(2n_{Q,h}) \exp(-E_g/\tau) = n_e^2.
		$$

		Since $n_e' = n_h'$ in equilibrium (which reflects that all electrons can interact with the holes),

		\begin{gather*}
		n_e'^2 - n_d n_e' - n_e^2 = 0 \\
		n_e = \frac{n_d + \sqrt{n_d^2  +4n_e^2}}{2} \\
		n_e = \frac{n_d}{2} + \sqrt{\frac{n_d^2}{4} + n_e^2}.
		\end{gather*}




	\end{enumerate}

\end{enumerate}

\end{document}